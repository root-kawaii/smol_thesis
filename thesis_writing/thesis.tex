% A LaTeX template for MSc Thesis submissions to 
% Politecnico di Milano (PoliMi) - School of Industrial and Information Engineering
%
% S. Bonetti, A. Gruttadauria, G. Mescolini, A. Zingaro
% e-mail: template-tesi-ingind@polimi.it
%
% Last Revision: October 2021
%
% Copyright 2021 Politecnico di Milano, Italy. NC-BY

\documentclass{Configuration_Files/PoliMi3i_thesis}

%------------------------------------------------------------------------------
%	REQUIRED PACKAGES AND  CONFIGURATIONS
%------------------------------------------------------------------------------

% CONFIGURATIONS
\usepackage{parskip} % For paragraph layout
\usepackage{setspace} % For using single or double spacing
\usepackage{emptypage} % To insert empty pages
\usepackage{multicol} % To write in multiple columns (executive summary)
\setlength\columnsep{15pt} % Column separation in executive summary
\setlength\parindent{0pt} % Indentation
\raggedbottom  

% PACKAGES FOR TITLES
\usepackage{titlesec}
% \titlespacing{\section}{left spacing}{before spacing}{after spacing}
\titlespacing{\section}{0pt}{3.3ex}{2ex}
\titlespacing{\subsection}{0pt}{3.3ex}{1.65ex}
\titlespacing{\subsubsection}{0pt}{3.3ex}{1ex}
\usepackage{color}

% PACKAGES FOR LANGUAGE AND FONT
\usepackage[english]{babel} % The document is in English  
\usepackage[utf8]{inputenc} % UTF8 encoding
\usepackage[T1]{fontenc} % Font encoding
\usepackage[11pt]{moresize} % Big fonts

% PACKAGES FOR IMAGES
\usepackage{graphicx}
\usepackage{transparent} % Enables transparent images
\usepackage{eso-pic} % For the background picture on the title page
\usepackage{subfig} % Numbered and caption subfigures using \subfloat.
\usepackage{tikz} % A package for high-quality hand-made figures.
\usetikzlibrary{}
\graphicspath{{./Images/}} % Directory of the images
\usepackage{caption} % Coloured captions
\usepackage{xcolor} % Coloured captions
\usepackage{amsthm,thmtools,xcolor} % Coloured "Theorem"
\usepackage{float}

% STANDARD MATH PACKAGES
\usepackage{amsmath}
\usepackage{amsthm}
\usepackage{amssymb}
\usepackage{amsfonts}
\usepackage{bm}
\usepackage[overload]{empheq} % For braced-style systems of equations.
\usepackage{fix-cm} % To override original LaTeX restrictions on sizes

% PACKAGES FOR TABLES
\usepackage{tabularx}
\usepackage{longtable} % Tables that can span several pages
\usepackage{colortbl}

% PACKAGES FOR ALGORITHMS (PSEUDO-CODE)
\usepackage{algorithm}
\usepackage{algorithmic}

% PACKAGES FOR REFERENCES & BIBLIOGRAPHY
\usepackage[colorlinks=true,linkcolor=black,anchorcolor=black,citecolor=black,filecolor=black,menucolor=black,runcolor=black,urlcolor=black]{hyperref} % Adds clickable links at references
\usepackage{cleveref}
\usepackage[square, numbers]{natbib} % Square brackets, citing references with numbers, citations sorted by appearance in the text and compressed
\bibliographystyle{abbrvnat} % You may use a different style adapted to your field

% OTHER PACKAGES
\usepackage{pdfpages} % To include a pdf file
\usepackage{afterpage}
\usepackage{lipsum} % DUMMY PACKAGE
\usepackage{fancyhdr} % For the headers
\fancyhf{}

% Input of configuration file. Do not change config.tex file unless you really know what you are doing. 
% Define blue color typical of polimi
\definecolor{bluepoli}{cmyk}{0.4,0.1,0,0.4}

% Custom theorem environments
\declaretheoremstyle[
  headfont=\color{bluepoli}\normalfont\bfseries,
  bodyfont=\color{black}\normalfont\itshape,
]{colored}

% Set-up caption colors
\captionsetup[figure]{labelfont={color=bluepoli}} % Set colour of the captions
\captionsetup[table]{labelfont={color=bluepoli}} % Set colour of the captions
\captionsetup[algorithm]{labelfont={color=bluepoli}} % Set colour of the captions

\theoremstyle{colored}
\newtheorem{theorem}{Theorem}[chapter]
\newtheorem{proposition}{Proposition}[chapter]

% Enhances the features of the standard "table" and "tabular" environments.
\newcommand\T{\rule{0pt}{2.6ex}}
\newcommand\B{\rule[-1.2ex]{0pt}{0pt}}

% Pseudo-code algorithm descriptions.
\newcounter{algsubstate}
\renewcommand{\thealgsubstate}{\alph{algsubstate}}
\newenvironment{algsubstates}
  {\setcounter{algsubstate}{0}%
   \renewcommand{\STATE}{%
     \stepcounter{algsubstate}%
     \Statex {\small\thealgsubstate:}\space}}
  {}

% New font size
\newcommand\numfontsize{\@setfontsize\Huge{200}{60}}

% Title format: chapter
\titleformat{\chapter}[hang]{
\fontsize{50}{20}\selectfont\bfseries\filright}{\textcolor{bluepoli} \thechapter\hsp\hspace{2mm}\textcolor{bluepoli}{|   }\hsp}{0pt}{\huge\bfseries \textcolor{bluepoli}
}

% Title format: section
\titleformat{\section}
{\color{bluepoli}\normalfont\Large\bfseries}
{\color{bluepoli}\thesection.}{1em}{}

% Title format: subsection
\titleformat{\subsection}
{\color{bluepoli}\normalfont\large\bfseries}
{\color{bluepoli}\thesubsection.}{1em}{}

% Title format: subsubsection
\titleformat{\subsubsection}
{\color{bluepoli}\normalfont\large\bfseries}
{\color{bluepoli}\thesubsubsection.}{1em}{}

% Shortening for setting no horizontal-spacing
\newcommand{\hsp}{\hspace{0pt}}

\makeatletter
% Renewcommand: cleardoublepage including the background pic
\renewcommand*\cleardoublepage{%
  \clearpage\if@twoside\ifodd\c@page\else
  \null
  \AddToShipoutPicture*{\BackgroundPic}
  \thispagestyle{empty}%
  \newpage
  \if@twocolumn\hbox{}\newpage\fi\fi\fi}
\makeatother

%For correctly numbering algorithms
\numberwithin{algorithm}{chapter}

%----------------------------------------------------------------------------
%	NEW COMMANDS DEFINED
%----------------------------------------------------------------------------

% EXAMPLES OF NEW COMMANDS
\newcommand{\bea}{\begin{eqnarray}} % Shortcut for equation arrays
\newcommand{\eea}{\end{eqnarray}}
\newcommand{\e}[1]{\times 10^{#1}}  % Powers of 10 notation

%----------------------------------------------------------------------------
%	ADD YOUR PACKAGES (be careful of package interaction)
%----------------------------------------------------------------------------

%----------------------------------------------------------------------------
%	ADD YOUR DEFINITIONS AND COMMANDS (be careful of existing commands)
%----------------------------------------------------------------------------

%----------------------------------------------------------------------------
%	BEGIN OF YOUR DOCUMENT
%----------------------------------------------------------------------------

\begin{document}

\fancypagestyle{plain}{%
\fancyhf{} % Clear all header and footer fields
\fancyhead[RO,RE]{\thepage} %RO=right odd, RE=right even
\renewcommand{\headrulewidth}{0pt}
\renewcommand{\footrulewidth}{0pt}}

%----------------------------------------------------------------------------
%	TITLE PAGE
%----------------------------------------------------------------------------

\pagestyle{empty} % No page numbers
\frontmatter % Use roman page numbering style (i, ii, iii, iv...) for the preamble pages

\puttitle{
	title=Title, % Title of the thesis
	name=Matteo Regge and Manuel Stoppiello, % Author Name and Surname
	course=Xxxxxxx Engineering - Ingegneria Xxxxxxx, % Study Programme (in Italian)
	ID  = 10619213,  % Student ID number (numero di matricola)
	advisor= Prof. Maurizio Magarini, % Supervisor name
	coadvisor={Antonio Coviello}, % Co-Supervisor name, remove this line if there is none
	academicyear={2023-24},  % Academic Year
} % These info will be put into your Title page 

%----------------------------------------------------------------------------
%	PREAMBLE PAGES: ABSTRACT (inglese e italiano), EXECUTIVE SUMMARY
%----------------------------------------------------------------------------
\startpreamble
\setcounter{page}{1} % Set page counter to 1

% ABSTRACT IN ENGLISH
\chapter*{Abstract} 
Here goes the Abstract in English of your thesis followed by a list of keywords.
The Abstract is a concise summary of the content of the thesis (single page of text)
and a guide to the most important contributions included in your thesis.
The Abstract is the very last thing you write.
It should be a self-contained text and should be clear to someone who hasn't (yet) read the whole manuscript.
The Abstract should contain the answers to the main scientific questions that have been addressed in your thesis.
It needs to summarize the adopted motivations and the adopted methodological approach as                          well as the findings of your work and their relevance and impact.
The Abstract is the part appearing in the record of your thesis inside POLITesi,
the Digital Archive of PhD and Master Theses (Laurea Magistrale) of Politecnico di Milano.
The Abstract will be followed by a list of four to six keywords.
Keywords are a tool to help indexers and search engines to find relevant documents.
To be relevant and effective, keywords must be chosen carefully.
They should represent the content of your work and be specific to your field or sub-field.
Keywords may be a single word or two to four words. 
\\
\\
\textbf{Keywords:} here, the keywords, of your thesis % Keywords

% ABSTRACT IN ITALIAN
\chapter*{Abstract in lingua italiana}
Qui va l'Abstract in lingua italiana della tesi seguito dalla lista di parole chiave.
\\
\\
\textbf{Parole chiave:} qui, vanno, le parole chiave, della tesi % Keywords (italian)

%----------------------------------------------------------------------------
%	LIST OF CONTENTS/FIGURES/TABLES/SYMBOLS
%----------------------------------------------------------------------------

% TABLE OF CONTENTS
\thispagestyle{empty}
\tableofcontents % Table of contents 
\thispagestyle{empty}
\cleardoublepage

%-------------------------------------------------------------------------
%	THESIS MAIN TEXT
%-------------------------------------------------------------------------
% In the main text of your thesis you can write the chapters in two different ways:
%
%(1) As presented in this template you can write:
%    \chapter{Title of the chapter}
%    *body of the chapter*
%
%(2) You can write your chapter in a separated .tex file and then include it in the main file with the following command:
%    \chapter{Title of the chapter}
%    \input{chapter_file.tex}
%
% Especially for long thesis, we recommend you the second option.

\addtocontents{toc}{\vspace{2em}} % Add a gap in the Contents, for aesthetics
\mainmatter % Begin numeric (1,2,3...) page numbering

% --------------------------------------------------------------------------
% NUMBERED CHAPTERS % Regular chapters following
% --------------------------------------------------------------------------
\chapter{Introduction}
\section{PNRelay project}
Peripheral nerve injuries (PNI) are mainly caused by surgery and trauma and are common in clinical practice, with 13 to 23 per 1,00,000 people typically suffering from PNI.
\cite{zhangResearchHotspotsTrends2022}
Traumatic injury to peripheral nerves is a significant cause of morbidity and disability today. Different types of extremity trauma can result in specific damage to particular nerves associated with that limb. Extremity trauma may be obvious as to its location, but the depth, severity, and underlying structures involved are not always clear.
\cite{taylorIncidencePeripheralNerve2008}
Common etiologies of acute traumatic peripheral nerve injury (TPNI) include penetrating injury, crush, stretch, and ischemia. Management of TPNI requires familiarity with the relevant anatomy, pathology, pathophysiology, and the surgical principles, approaches and concerns.
\cite{campbellEvaluationManagementPeripheral2008}
The PNRelay project has been presented as a collaboration effort between the Politecnico di Milano and the Politecnico di Torino.
The objective of the project is to target this problem developing a new peripheral nerve interface, capable of conveying information between brain and organs. This kind of medical devices generally record the electrical stimulus travelling within the nerve, analyze it and finally artificially stimulate the nerve below the injury to replicate the natural response of the body.\cite{garbelliniPNRELAY2020}
The device that has been engineered as part of the PNRelay is made of two main subparts:
one that is implanted and one that is external.
\\
% copied from davide
The internal implant includes cuff electrodes for signal acquisition, the Senseback ASIC chip [19] for signal processing, enclosed within a biocompatible capsule, and a transdermal port for wireless device powering. Extra-neural cuff electrodes, positioned externally on the nerve’s surface, are chosen for their lower invasiveness in acquiring ENG data [20]. The data transmission module, which is responsible for data signal processing, is currently being developed for full-body integration. The external component hosts the classifier, which by performing an online operation, allows the different stimuli to be recognized. The signal is then sent to the stimulator, which proceeds to close the loop. The overall scheme for animal testing is shown in Figure 1.1. During initial experimental phases, the classifier may operate on a computer connected to the external circuit, with the ultimate objective being its integration into the implanted circuitry. Due to the typically substan- tial variations in neurological data among individuals, classification algorithms in this domain are often tailored to specific subjects [3]. This further compounds the challenge of gathering a sufficient amount of biological data on same subject. Challenges persist in decoding neural signal information due to limitations imposed by acquisition invasiveness. While extra-neural cuff electrodes represent a less invasive option for chronic implantation [21], the classification of sensory stimuli recorded through them remains complex due to a limited SNR. Various techniques have been explored to address this challenge [22]–[24], but the optimal classification approach remains an open area for further research and exploration.
% copied from davide
\\
\subsection{PNRelay Setup}

\subsection{Ethical concerns}


\chapter{Background}

\section{Norms}
\section{Nervous Systems}

\section{Signal Acquisition}



\chapter{State of the Art}



\section{Introduction to Bluetooth Low Energy}

Firstly, we need to define that "Bluetooth low energy is a brand new technology that has been designed as both a complementary
technology to classic Bluetooth as well as the lowest possible power wireless technology that can be
designed and built. Although it uses the Bluetooth brand and borrows a lot of technology from its
parent, Bluetooth low energy should be considered a different technology, addressing different design
goals". \cite{heydonBluetoothLowEnergy2015}
""chiara""
The main advantage of the BLE over the
standard Bluetooth is that it keeps the radio off as much as possible when no data has
to be sent [47]. It is optimized for low power consumption and short-range
communication; thus, it enables devices to operate on a small energy budget, making
Bluetooth technology is based on a master-slave concept, where one device acts as the
master and controls one or more slave devices. The master initiates and maintains the
communication, while the slave devices respond to the master's commands. For this
there are four roles that a BLE device can implement. A device can be a peripheral
(slave), it advertises their presence and respond to central devices' requests. A central
(master), instead, scans for these advertising packets and it could, if the advertisement
packet allows it, connect to that device. Moreover, there are other two roles possible:
the broadcaster and the observer. A broadcaster sends out advertising packets without
allowing any connections, while an observer discovers peripherals and broadcasters,
but without the capability of accepting connections from a central.
""chiara""

\section{Communication system}
\section{Chiara code}
\section{Communication speed}
\section{ENGNet advantages and shortcomings}

EEGNet was introduced by [27] as a variant of CNN specifically designed for Brain- Computer Interace (BCI) applications. Its key advantage lies in being a compact ar- chitecture with fewer parameters compared to traditional CNNs. This makes EEGNet well-suited for processing EEG signals efficiently and effectively, enabling improved per- formance in BCI tasks while reducing computational complexity. EEGNet has also been tested for Steady-state Visual Evoked Potentials (SSVEP) signals, a type of EEG response elicited by visual stimulation [28]. These tests have demonstrated the versatility of EEG- Net in handling various EEG-based BCI tasks, showing promising results in accurately classifying SSVEP responses and opening up new possibilities for real-world applications in brain-computer communication and control systems.

\section{ENGNet architecture}

The architecture involves two convolutional steps, employing 2D convolutional filters to capture EEG signals at different band-pass frequencies and depthwise convolutions to learn spatial filters for each temporal filter. The approach also utilizes separable con- volutions to reduce the number of parameters and efficiently combine feature maps. In the classification block, softmax classification is directly applied to the extracted features without a dense layer, reducing the number of parameters in the model.


\section{ENGNet layers}

'foto'

\section{ENGNet results}
Analysis of weights in the first convolutional layer indicate the network’s capability to recognize and accentuate characteristic ENG signal frequencies, and these are relevant to improves classification performance. However, the network’s ability to identify these frequencies depends on the quality of the signal used during training. The dependence of the power spectrum analysis on the data suggests the importance of carrying out specific training per subject, given that even within data taken from the same animal there is a variation in the frequencies emphasized by ENGNet,

Talk about limitation

Comparison between validation and test re- sults showed that ENGNet can generalize well to new data. This can be further improved by increasing the dropout probability


'foto dei results'



\section{nrf5280 Board pros and cons}
\section{Power Supplier}
\section{Memory management}
\section{Radiation absorbtion}
\section{Channel selection}

We have already described in previous sections the project at hand and why every possible improvement can ease the development of a functional implementation.
One of those crucial aspects is the amount of information sent between the devices, in the particular the channels that get transmitted.
As previously shown, the dataset at hand was registered using cuff electrodes chirurgically implanted on the animals' peripheral nerves; each of these cuff electrodes has 16 sensors that are able to measure the voltage of the nerve they cover.
Each of these sensors, being in a different position, will register different data, furthermore one or more of these sensors may be damaged during the surgery or may record data with high noise.
Because of this, not only it would be useful to limit the channel we share between devices to help with the communication, but we could also help the classifier by removing channels with faulty or noisy sensors.
In short, the three benefits of channel selection are : reducing the computational complexity of any process on the dataset, reduce the amount of overfitting of the models and reudce the times to setup the application. \cite{alotaibyReviewChannelSelection2015}
There is a great amount of literature that deals with the issues of channel selection of EEG channels, since it is a well known area of research and proves to be very helpful in reaching better results.
In particular our area of interest focused on channel selection applied to motion-imagery EEG Signals (MI-EEG).
We wanted to find algorithm that not only performed well on the usual Accuracy-F1Score metrics while reducing the numbers of channels, but also would fit well with our models.
Given the architecture of our ENGNet model, we needed an algorithm to perform channel selectiom that would work with a Neural Network model.
''' expand test case scenario and explain why '''
In addition, given future test case scenarios we also needed an algorithm that could be performed in an "on-line" way, or as close to it as possible.
To summarize we needed:
''list''
-online
-Cnn ready
-good performance on f1-Accuracy
-works well with few channels
''list''

There are many algorithms that reduce noise...  \cite{abdullahEEGChannelSelection2022}
but those are already inside the ENGNet since spatial features ...
Talk about why ASR didn't improve, basically same reason.
Ideally we would select channels, by iterating over all possible subsets of features but too much time.
Particle swarm optimization no, beacuse too much time.
Therefore correlation algorithms, promising and tested with CNN.
Since we already had CNN why not implement a new algorithm that hasn't been well tested that requires to add one layer.
Critical issue could be that generally black but we will discsuss it later.


\chapter{Conclusions and future developments}
\label{ch:conclusions}%
A final chapter containing the main conclusions of your research/study
and possible future developments of your work have to be inserted in this chapter.


%-------------------------------------------------------------------------
%	BIBLIOGRAPHY
%-------------------------------------------------------------------------

\addtocontents{toc}{\vspace{2em}} % Add a gap in the Contents, for aesthetics
\bibliography{Thesis_bibliography} % The references information are stored in the file named "Thesis_bibliography.bib"

%-------------------------------------------------------------------------
%	APPENDICES
%-------------------------------------------------------------------------

\cleardoublepage
\addtocontents{toc}{\vspace{2em}} % Add a gap in the Contents, for aesthetics
\appendix
\chapter{Appendix A}
If you need to include an appendix to support the research in your thesis, you can place it at the end of the manuscript.
An appendix contains supplementary material (figures, tables, data, codes, mathematical proofs, surveys, \dots)
which supplement the main results contained in the previous chapters.

\chapter{Appendix B}
It may be necessary to include another appendix to better organize the presentation of supplementary material.


% LIST OF FIGURES
\listoffigures

% LIST OF TABLES
\listoftables

% LIST OF SYMBOLS
% Write out the List of Symbols in this page
\chapter*{List of Symbols} % You have to include a chapter for your list of symbols (
\begin{table}[H]
    \centering
    \begin{tabular}{lll}
        \textbf{Variable} & \textbf{Description} & \textbf{SI unit} \\\hline\\[-9px]
        $\bm{u}$ & solid displacement & m \\[2px]
        $\bm{u}_f$ & fluid displacement & m \\[2px]
    \end{tabular}
\end{table}

% ACKNOWLEDGEMENTS
\chapter*{Acknowledgements}
Here you might want to .


\cleardoublepage

\end{document}
